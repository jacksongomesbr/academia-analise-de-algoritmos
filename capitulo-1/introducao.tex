
% Default to the notebook output style

    


% Inherit from the specified cell style.




    
\documentclass{article}

    
    
    \usepackage{graphicx} % Used to insert images
    \usepackage{adjustbox} % Used to constrain images to a maximum size 
    \usepackage{color} % Allow colors to be defined
    \usepackage{enumerate} % Needed for markdown enumerations to work
    \usepackage{geometry} % Used to adjust the document margins
    \usepackage{amsmath} % Equations
    \usepackage{amssymb} % Equations
    \usepackage[mathletters]{ucs} % Extended unicode (utf-8) support
    \usepackage[utf8x]{inputenc} % Allow utf-8 characters in the tex document
    \usepackage{fancyvrb} % verbatim replacement that allows latex
    \usepackage{grffile} % extends the file name processing of package graphics 
                         % to support a larger range 
    % The hyperref package gives us a pdf with properly built
    % internal navigation ('pdf bookmarks' for the table of contents,
    % internal cross-reference links, web links for URLs, etc.)
    \usepackage{hyperref}
    \usepackage{longtable} % longtable support required by pandoc >1.10
    \usepackage{booktabs}  % table support for pandoc > 1.12.2
    

    
    
    \definecolor{orange}{cmyk}{0,0.4,0.8,0.2}
    \definecolor{darkorange}{rgb}{.71,0.21,0.01}
    \definecolor{darkgreen}{rgb}{.12,.54,.11}
    \definecolor{myteal}{rgb}{.26, .44, .56}
    \definecolor{gray}{gray}{0.45}
    \definecolor{lightgray}{gray}{.95}
    \definecolor{mediumgray}{gray}{.8}
    \definecolor{inputbackground}{rgb}{.95, .95, .85}
    \definecolor{outputbackground}{rgb}{.95, .95, .95}
    \definecolor{traceback}{rgb}{1, .95, .95}
    % ansi colors
    \definecolor{red}{rgb}{.6,0,0}
    \definecolor{green}{rgb}{0,.65,0}
    \definecolor{brown}{rgb}{0.6,0.6,0}
    \definecolor{blue}{rgb}{0,.145,.698}
    \definecolor{purple}{rgb}{.698,.145,.698}
    \definecolor{cyan}{rgb}{0,.698,.698}
    \definecolor{lightgray}{gray}{0.5}
    
    % bright ansi colors
    \definecolor{darkgray}{gray}{0.25}
    \definecolor{lightred}{rgb}{1.0,0.39,0.28}
    \definecolor{lightgreen}{rgb}{0.48,0.99,0.0}
    \definecolor{lightblue}{rgb}{0.53,0.81,0.92}
    \definecolor{lightpurple}{rgb}{0.87,0.63,0.87}
    \definecolor{lightcyan}{rgb}{0.5,1.0,0.83}
    
    % commands and environments needed by pandoc snippets
    % extracted from the output of `pandoc -s`
    \DefineVerbatimEnvironment{Highlighting}{Verbatim}{commandchars=\\\{\}}
    % Add ',fontsize=\small' for more characters per line
    \newenvironment{Shaded}{}{}
    \newcommand{\KeywordTok}[1]{\textcolor[rgb]{0.00,0.44,0.13}{\textbf{{#1}}}}
    \newcommand{\DataTypeTok}[1]{\textcolor[rgb]{0.56,0.13,0.00}{{#1}}}
    \newcommand{\DecValTok}[1]{\textcolor[rgb]{0.25,0.63,0.44}{{#1}}}
    \newcommand{\BaseNTok}[1]{\textcolor[rgb]{0.25,0.63,0.44}{{#1}}}
    \newcommand{\FloatTok}[1]{\textcolor[rgb]{0.25,0.63,0.44}{{#1}}}
    \newcommand{\CharTok}[1]{\textcolor[rgb]{0.25,0.44,0.63}{{#1}}}
    \newcommand{\StringTok}[1]{\textcolor[rgb]{0.25,0.44,0.63}{{#1}}}
    \newcommand{\CommentTok}[1]{\textcolor[rgb]{0.38,0.63,0.69}{\textit{{#1}}}}
    \newcommand{\OtherTok}[1]{\textcolor[rgb]{0.00,0.44,0.13}{{#1}}}
    \newcommand{\AlertTok}[1]{\textcolor[rgb]{1.00,0.00,0.00}{\textbf{{#1}}}}
    \newcommand{\FunctionTok}[1]{\textcolor[rgb]{0.02,0.16,0.49}{{#1}}}
    \newcommand{\RegionMarkerTok}[1]{{#1}}
    \newcommand{\ErrorTok}[1]{\textcolor[rgb]{1.00,0.00,0.00}{\textbf{{#1}}}}
    \newcommand{\NormalTok}[1]{{#1}}
    
    % Define a nice break command that doesn't care if a line doesn't already
    % exist.
    \def\br{\hspace*{\fill} \\* }
    % Math Jax compatability definitions
    \def\gt{>}
    \def\lt{<}
    % Document parameters
    \title{introducao}
    
    
    

    % Pygments definitions
    
\makeatletter
\def\PY@reset{\let\PY@it=\relax \let\PY@bf=\relax%
    \let\PY@ul=\relax \let\PY@tc=\relax%
    \let\PY@bc=\relax \let\PY@ff=\relax}
\def\PY@tok#1{\csname PY@tok@#1\endcsname}
\def\PY@toks#1+{\ifx\relax#1\empty\else%
    \PY@tok{#1}\expandafter\PY@toks\fi}
\def\PY@do#1{\PY@bc{\PY@tc{\PY@ul{%
    \PY@it{\PY@bf{\PY@ff{#1}}}}}}}
\def\PY#1#2{\PY@reset\PY@toks#1+\relax+\PY@do{#2}}

\expandafter\def\csname PY@tok@gd\endcsname{\def\PY@tc##1{\textcolor[rgb]{0.63,0.00,0.00}{##1}}}
\expandafter\def\csname PY@tok@gu\endcsname{\let\PY@bf=\textbf\def\PY@tc##1{\textcolor[rgb]{0.50,0.00,0.50}{##1}}}
\expandafter\def\csname PY@tok@gt\endcsname{\def\PY@tc##1{\textcolor[rgb]{0.00,0.27,0.87}{##1}}}
\expandafter\def\csname PY@tok@gs\endcsname{\let\PY@bf=\textbf}
\expandafter\def\csname PY@tok@gr\endcsname{\def\PY@tc##1{\textcolor[rgb]{1.00,0.00,0.00}{##1}}}
\expandafter\def\csname PY@tok@cm\endcsname{\let\PY@it=\textit\def\PY@tc##1{\textcolor[rgb]{0.25,0.50,0.50}{##1}}}
\expandafter\def\csname PY@tok@vg\endcsname{\def\PY@tc##1{\textcolor[rgb]{0.10,0.09,0.49}{##1}}}
\expandafter\def\csname PY@tok@m\endcsname{\def\PY@tc##1{\textcolor[rgb]{0.40,0.40,0.40}{##1}}}
\expandafter\def\csname PY@tok@mh\endcsname{\def\PY@tc##1{\textcolor[rgb]{0.40,0.40,0.40}{##1}}}
\expandafter\def\csname PY@tok@go\endcsname{\def\PY@tc##1{\textcolor[rgb]{0.53,0.53,0.53}{##1}}}
\expandafter\def\csname PY@tok@ge\endcsname{\let\PY@it=\textit}
\expandafter\def\csname PY@tok@vc\endcsname{\def\PY@tc##1{\textcolor[rgb]{0.10,0.09,0.49}{##1}}}
\expandafter\def\csname PY@tok@il\endcsname{\def\PY@tc##1{\textcolor[rgb]{0.40,0.40,0.40}{##1}}}
\expandafter\def\csname PY@tok@cs\endcsname{\let\PY@it=\textit\def\PY@tc##1{\textcolor[rgb]{0.25,0.50,0.50}{##1}}}
\expandafter\def\csname PY@tok@cp\endcsname{\def\PY@tc##1{\textcolor[rgb]{0.74,0.48,0.00}{##1}}}
\expandafter\def\csname PY@tok@gi\endcsname{\def\PY@tc##1{\textcolor[rgb]{0.00,0.63,0.00}{##1}}}
\expandafter\def\csname PY@tok@gh\endcsname{\let\PY@bf=\textbf\def\PY@tc##1{\textcolor[rgb]{0.00,0.00,0.50}{##1}}}
\expandafter\def\csname PY@tok@ni\endcsname{\let\PY@bf=\textbf\def\PY@tc##1{\textcolor[rgb]{0.60,0.60,0.60}{##1}}}
\expandafter\def\csname PY@tok@nl\endcsname{\def\PY@tc##1{\textcolor[rgb]{0.63,0.63,0.00}{##1}}}
\expandafter\def\csname PY@tok@nn\endcsname{\let\PY@bf=\textbf\def\PY@tc##1{\textcolor[rgb]{0.00,0.00,1.00}{##1}}}
\expandafter\def\csname PY@tok@no\endcsname{\def\PY@tc##1{\textcolor[rgb]{0.53,0.00,0.00}{##1}}}
\expandafter\def\csname PY@tok@na\endcsname{\def\PY@tc##1{\textcolor[rgb]{0.49,0.56,0.16}{##1}}}
\expandafter\def\csname PY@tok@nb\endcsname{\def\PY@tc##1{\textcolor[rgb]{0.00,0.50,0.00}{##1}}}
\expandafter\def\csname PY@tok@nc\endcsname{\let\PY@bf=\textbf\def\PY@tc##1{\textcolor[rgb]{0.00,0.00,1.00}{##1}}}
\expandafter\def\csname PY@tok@nd\endcsname{\def\PY@tc##1{\textcolor[rgb]{0.67,0.13,1.00}{##1}}}
\expandafter\def\csname PY@tok@ne\endcsname{\let\PY@bf=\textbf\def\PY@tc##1{\textcolor[rgb]{0.82,0.25,0.23}{##1}}}
\expandafter\def\csname PY@tok@nf\endcsname{\def\PY@tc##1{\textcolor[rgb]{0.00,0.00,1.00}{##1}}}
\expandafter\def\csname PY@tok@si\endcsname{\let\PY@bf=\textbf\def\PY@tc##1{\textcolor[rgb]{0.73,0.40,0.53}{##1}}}
\expandafter\def\csname PY@tok@s2\endcsname{\def\PY@tc##1{\textcolor[rgb]{0.73,0.13,0.13}{##1}}}
\expandafter\def\csname PY@tok@vi\endcsname{\def\PY@tc##1{\textcolor[rgb]{0.10,0.09,0.49}{##1}}}
\expandafter\def\csname PY@tok@nt\endcsname{\let\PY@bf=\textbf\def\PY@tc##1{\textcolor[rgb]{0.00,0.50,0.00}{##1}}}
\expandafter\def\csname PY@tok@nv\endcsname{\def\PY@tc##1{\textcolor[rgb]{0.10,0.09,0.49}{##1}}}
\expandafter\def\csname PY@tok@s1\endcsname{\def\PY@tc##1{\textcolor[rgb]{0.73,0.13,0.13}{##1}}}
\expandafter\def\csname PY@tok@kd\endcsname{\let\PY@bf=\textbf\def\PY@tc##1{\textcolor[rgb]{0.00,0.50,0.00}{##1}}}
\expandafter\def\csname PY@tok@sh\endcsname{\def\PY@tc##1{\textcolor[rgb]{0.73,0.13,0.13}{##1}}}
\expandafter\def\csname PY@tok@sc\endcsname{\def\PY@tc##1{\textcolor[rgb]{0.73,0.13,0.13}{##1}}}
\expandafter\def\csname PY@tok@sx\endcsname{\def\PY@tc##1{\textcolor[rgb]{0.00,0.50,0.00}{##1}}}
\expandafter\def\csname PY@tok@bp\endcsname{\def\PY@tc##1{\textcolor[rgb]{0.00,0.50,0.00}{##1}}}
\expandafter\def\csname PY@tok@c1\endcsname{\let\PY@it=\textit\def\PY@tc##1{\textcolor[rgb]{0.25,0.50,0.50}{##1}}}
\expandafter\def\csname PY@tok@kc\endcsname{\let\PY@bf=\textbf\def\PY@tc##1{\textcolor[rgb]{0.00,0.50,0.00}{##1}}}
\expandafter\def\csname PY@tok@c\endcsname{\let\PY@it=\textit\def\PY@tc##1{\textcolor[rgb]{0.25,0.50,0.50}{##1}}}
\expandafter\def\csname PY@tok@mf\endcsname{\def\PY@tc##1{\textcolor[rgb]{0.40,0.40,0.40}{##1}}}
\expandafter\def\csname PY@tok@err\endcsname{\def\PY@bc##1{\setlength{\fboxsep}{0pt}\fcolorbox[rgb]{1.00,0.00,0.00}{1,1,1}{\strut ##1}}}
\expandafter\def\csname PY@tok@mb\endcsname{\def\PY@tc##1{\textcolor[rgb]{0.40,0.40,0.40}{##1}}}
\expandafter\def\csname PY@tok@ss\endcsname{\def\PY@tc##1{\textcolor[rgb]{0.10,0.09,0.49}{##1}}}
\expandafter\def\csname PY@tok@sr\endcsname{\def\PY@tc##1{\textcolor[rgb]{0.73,0.40,0.53}{##1}}}
\expandafter\def\csname PY@tok@mo\endcsname{\def\PY@tc##1{\textcolor[rgb]{0.40,0.40,0.40}{##1}}}
\expandafter\def\csname PY@tok@kn\endcsname{\let\PY@bf=\textbf\def\PY@tc##1{\textcolor[rgb]{0.00,0.50,0.00}{##1}}}
\expandafter\def\csname PY@tok@mi\endcsname{\def\PY@tc##1{\textcolor[rgb]{0.40,0.40,0.40}{##1}}}
\expandafter\def\csname PY@tok@gp\endcsname{\let\PY@bf=\textbf\def\PY@tc##1{\textcolor[rgb]{0.00,0.00,0.50}{##1}}}
\expandafter\def\csname PY@tok@o\endcsname{\def\PY@tc##1{\textcolor[rgb]{0.40,0.40,0.40}{##1}}}
\expandafter\def\csname PY@tok@kr\endcsname{\let\PY@bf=\textbf\def\PY@tc##1{\textcolor[rgb]{0.00,0.50,0.00}{##1}}}
\expandafter\def\csname PY@tok@s\endcsname{\def\PY@tc##1{\textcolor[rgb]{0.73,0.13,0.13}{##1}}}
\expandafter\def\csname PY@tok@kp\endcsname{\def\PY@tc##1{\textcolor[rgb]{0.00,0.50,0.00}{##1}}}
\expandafter\def\csname PY@tok@w\endcsname{\def\PY@tc##1{\textcolor[rgb]{0.73,0.73,0.73}{##1}}}
\expandafter\def\csname PY@tok@kt\endcsname{\def\PY@tc##1{\textcolor[rgb]{0.69,0.00,0.25}{##1}}}
\expandafter\def\csname PY@tok@ow\endcsname{\let\PY@bf=\textbf\def\PY@tc##1{\textcolor[rgb]{0.67,0.13,1.00}{##1}}}
\expandafter\def\csname PY@tok@sb\endcsname{\def\PY@tc##1{\textcolor[rgb]{0.73,0.13,0.13}{##1}}}
\expandafter\def\csname PY@tok@k\endcsname{\let\PY@bf=\textbf\def\PY@tc##1{\textcolor[rgb]{0.00,0.50,0.00}{##1}}}
\expandafter\def\csname PY@tok@se\endcsname{\let\PY@bf=\textbf\def\PY@tc##1{\textcolor[rgb]{0.73,0.40,0.13}{##1}}}
\expandafter\def\csname PY@tok@sd\endcsname{\let\PY@it=\textit\def\PY@tc##1{\textcolor[rgb]{0.73,0.13,0.13}{##1}}}

\def\PYZbs{\char`\\}
\def\PYZus{\char`\_}
\def\PYZob{\char`\{}
\def\PYZcb{\char`\}}
\def\PYZca{\char`\^}
\def\PYZam{\char`\&}
\def\PYZlt{\char`\<}
\def\PYZgt{\char`\>}
\def\PYZsh{\char`\#}
\def\PYZpc{\char`\%}
\def\PYZdl{\char`\$}
\def\PYZhy{\char`\-}
\def\PYZsq{\char`\'}
\def\PYZdq{\char`\"}
\def\PYZti{\char`\~}
% for compatibility with earlier versions
\def\PYZat{@}
\def\PYZlb{[}
\def\PYZrb{]}
\makeatother


    % Exact colors from NB
    \definecolor{incolor}{rgb}{0.0, 0.0, 0.5}
    \definecolor{outcolor}{rgb}{0.545, 0.0, 0.0}



    
    % Prevent overflowing lines due to hard-to-break entities
    \sloppy 
    % Setup hyperref package
    \hypersetup{
      breaklinks=true,  % so long urls are correctly broken across lines
      colorlinks=true,
      urlcolor=blue,
      linkcolor=darkorange,
      citecolor=darkgreen,
      }
    % Slightly bigger margins than the latex defaults
    
    \geometry{verbose,tmargin=1in,bmargin=1in,lmargin=1in,rmargin=1in}
    
    

    \begin{document}
    
    
    \maketitle
    
    

    

    \section{Introdução}


    O entendimento da complexidade de algoritmos é uma tarefa chave no
desenvolvimento de sistemas e algoritmos capazes não apenas de fornecer
a saída correta, mas de acordo com restrições impostas pelo contexto ou
pelo domínio. Para exemplificar esta afirmação, tomemos como exemplo um
algoritmo que precisa encontrar um número \(n\) em um vetor \(D\). O
algoritmo deve retornar um par (\(a\), \(b\)) onde:

\begin{itemize}
\itemsep1pt\parskip0pt\parsep0pt
\item
  \(a\) : Um valor booleano indicando se o número \(n\) foi encontrado;
  e
\item
  \(b\) : Um número inteiro indicando a posição de \(n\) no vetor.
\end{itemize}

Uma possível implementação deste algoritmo é a seguinte (em
pseudo-código).

\textbf{Algoritmo 1.1:}

    \begin{Shaded}
\begin{Highlighting}[]
\DataTypeTok{int} \NormalTok{i = }\DecValTok{0}\NormalTok{; }
\FunctionTok{foreach} \NormalTok{(d in D) \{}
      \KeywordTok{if} \NormalTok{(d == n) \{}
          \KeywordTok{return} \NormalTok{\{True, d\};}
      \NormalTok{\}}
      \NormalTok{i++;}
\NormalTok{\}}
\KeywordTok{return} \NormalTok{\{False, }\DecValTok{0}\NormalTok{\};}
\end{Highlighting}
\end{Shaded}

    Chama-se \textbf{operação fundamental} a operação principal do
algoritmo. Neste caso, digamos que a operação fundamental é a igualdade
(comparação) da linha 3.

É possível perceber que a quantidade de \textbf{operações fundamentais}
varia conforme o tamanho de \(D\) (\(|D|\)). Podemos analisar das
seguintes formas:

\begin{enumerate}
\def\labelenumi{\arabic{enumi}.}
\itemsep1pt\parskip0pt\parsep0pt
\item
  Considerando \(n \in D\):

  \begin{enumerate}
  \def\labelenumii{\arabic{enumii}.}
  \itemsep1pt\parskip0pt\parsep0pt
  \item
    Se \(|D| = 1\), então a operação fundamental será executada 1 vez; e
  \item
    Se \(|D| = 10\), então a operação fundamental será executada, no
    máximo, 10 vezes.
  \end{enumerate}
\item
  Considerando \(n \not\in D\):

  \begin{enumerate}
  \def\labelenumii{\arabic{enumii}.}
  \itemsep1pt\parskip0pt\parsep0pt
  \item
    Independentemente \(|D|\), a operação fundamental será executada
    \(|D|\) vezes.
  \end{enumerate}
\end{enumerate}

** Exercício 1.1:** Considerando as afirmações anteriores, qual a
diferença entre \(n\) estar no início ou no final de \(D\)? Demonstre e
justifique a afirmação 1B.

A análise destes casos e a representação disso seguindo um formalismo é
o que chamamos \textbf{Análise de Algoritmos}. Existem diversas formas
de analisar algoritmos, ou a computação deles, como considerar o tempo
de execução e o espaço (a memória necessária para a execução do
algoritmo). Neste curso estamos mais interessados na primeira forma, a
análise do tempo de execução.

De modo geral, a análise do comportamento de algoritmos é estudada pela
\textbf{Complexidade de Algoritmos}. Portanto, \textbf{Complexidade de
Algoritmos} é a análise do esforço computacional (tempo ou memória)
necessário para executar um algoritmo.

É importante ter a noção de que, para um determinado problema, podem
haver diversas implementações.

** Exercício 1.2:** Este exercício será feito em grupo. Cada grupo deve
escolher uma linguagem ou plataforma diferente (ex.: Java, C++, C\#,
Javascript, Python etc.). Seu programa deverá ler um arquivo com o
seguinte formato:

\begin{itemize}
\itemsep1pt\parskip0pt\parsep0pt
\item
  A primeira linha contém o número \(n\);
\item
  A segunda linha possum um número inteiro que corresponde à quantidade
  de números do conjunto \(D\); e
\item
  Da terceira linha em diante, estão os números do conjunto \(D\).
\end{itemize}

Após ler o arquivo, seu programa deve procurar o número \(n\) no
conjunto \(D\) e gerar um arquivo texto contendo três números, um em
cada linha:

\begin{itemize}
\itemsep1pt\parskip0pt\parsep0pt
\item
  A palavra True ou False, conforme a busca de \(n\) em \(D\);
\item
  Um número inteiro que corresponde à posição de \(n\) em \(D\). Se
  \(n \not\in D\), então o valor pode ser \(0\) (zero); e
\item
  Um número real que corresponde ao tempo de execução do programa.
\end{itemize}

Para verificar corretitude do seu programa, você deve usar os três
arquivos a seguir:

\begin{enumerate}
\def\labelenumi{\arabic{enumi}.}
\itemsep1pt\parskip0pt\parsep0pt
\item
  \url{dataset-1-a.csv}
\item
  \url{dataset-1-b.csv}
\item
  \url{dataset-1-c.csv}
\end{enumerate}


    \subsection{Complexidade e Desempenho de Algoritmos}


    Como você pode perceber, o tempo de execução do algoritmo pode variar
conforme questões como a plataforma de programação e o ambiente de
execução (a máquina, em si). É importante entender que este formato não
pode ser usado como ferramenta para uma análise mais criteriosa e geral
de algoritmos. Ainda, podemos dizer que a execução do algoritmo depende
do conjunto de entradas e da sequência de operações fundamentais
necessárias para o seu funcionamento. Portanto, vamos a alguns
formalismos:

\begin{itemize}
\itemsep1pt\parskip0pt\parsep0pt
\item
  \(D\) é conjunto de dados de entrada do algoritmo; e
\item
  \(E\) é o conjunto de operações fundamentais para o funcionamento do
  algoritmo.
\end{itemize}

Assim, podemos definir um algoritmo como a função
\(a : D \rightarrow E\).

\textbf{Execução} resulta a sequência de execuções de operações
fundamentais realizadas durante a execução do algoritmo \(a\). É
representada como \(exec(a) : D \rightarrow E\).

\textbf{Custo} dá o comprimento da sequência \(e \in E\). É representado
como \(custo : E \rightarrow \mathbb{R}_+\). Ou seja, \textbf{custo} é
um real positivo que representa a quantidade de operações fundamentais.
Obviamente, quanto menor a quantidade de operações, menor o custo.

\textbf{Desempenho} dá o custo (em termos das operações fundamentais) da
execução de \(a\) com a entrada \(d \in D\). É representado como
\(desempenho(a, d) := custo( exec(a, d) )\). Isso nos permite analisar
um algoritmo considerando as diversas entradas possíveis e, portanto, o
desempenho de um algoritmo \(a\) com uma entrada \(d \in D\) mede o
custo da execução do algoritmo sobre esta entrada.

Se você preferir, também pode usar a notação a seguir:

\begin{itemize}
\itemsep1pt\parskip0pt\parsep0pt
\item
  Algoritmo: \(a(d) = e\). Ou seja, um algoritmo \(a\) é uma função que
  recebe uma entrada \(d\) e resulta em uma sequência de operações
  fundamentais \(e\).
\end{itemize}

É importante perceber que o conceito de \emph{desempenho} indica a
quantidade de operações fundamentais executadas pelo algoritmo quando a
entrada é \(d\). Portanto, não se pode, necessariamente, julgar o
algoritmo como melhor ou pior com base nesta quantiade. Um elemento é o
fator \emph{tempo}. Considerando, por exemplo, dois algoritmos: \(a_1\)
e \(a_2\). Ambos executam sobre a mesma entrada \(d\). O algoritmo
\(a_1\) executa 10 operações fundamentais em 1 segundo. O algoritmo
\(a_2\) executa a mesma quantidade de operações em 0.5 segundos.
Portanto, o conceito de \emph{desempenho} visto até o momento é
dissociado do fator \emph{tempo}. Mais detalhes sobre isso serão vistos
posteriormente.

\textbf{Exercício 1.3}: O desempenho de um algoritmo sempre cresce com o
tamanho da entrada? Por quê?

\textbf{Exercício 1.4}: Considere o problema de encontrar o maior valor
em um conjunto de dados. Conside, também, que haja diversos conjuntos de
dados, em arquivos texto que contêm os elementos destes conjuntos, um em
cada linha. Crie um programa que lê cada um dos conjuntos de dados a
seguir, procura o maior valor e gera um arquivo de saída contendo, em
cada linha: o maior valor encontrado e o tempo de execução do algoritmo.
Depois de executar estes experimentos, lendo os arquivos de dados e
encontrando o maior valor, plote os tempos de execução um gráfico que
representa a evolução do desempenho (a execução do algoritmo conforme a
entrada). O que se pode perceber? Comente.

Os conjuntos de dados:

\begin{itemize}
\itemsep1pt\parskip0pt\parsep0pt
\item
  \url{dataset-2-a.csv}
\item
  \url{dataset-2-b.csv}
\item
  \url{dataset-2-c.csv}
\item
  \url{dataset-2-d.csv}
\item
  \url{dataset-2-e.csv} ou sua versão compactada (com
  \textasciitilde{}3,8MB): \url{dataset-2-e.rar}
\end{itemize}


    \subsection{Complexidade de algoritmo}


    Embora a função \(complexidade()\) vista anteriormente seja útil como um
critério para se entender a complexidade de algoritmos, ela não é
suficiente. A análise do algoritmo 1.1 demonstrou justamente isso: a
quantidade de operações fundamentais varia conforme a entrada e,
portanto, a noção atual de desempenho precisa ser complementada.

A função \textbf{tamanho} é definida como: \(tamanho(d) = n\), com
\(d \in D\). Esta função retorna \(n\): o tamanho (a quantidade de
elementos) da entrada \(d\).

No caso de a entrada ser uma lista, \(tamanho(d)\) resulta na quantidade
de elementos da lista. No caso de ser um grafo, a informação pode ser
composta pela quantidade de vértices e arestas.

\textbf{Exercício 1.5:} Por que é importante a utilização da função
\(tamanho()\) no estudo da complexidade de um algoritmo?

Utilizando a função \(tamanho(d)\), a função \(desempenho(a, d)\) pode
ser condensada na função \textbf{avalia}, definida como:
\(avalia(a, n)\) resulta em um número real que representa uma medida de
complexidade do algoritmo com base no tamanho da entrada. Este valor
pode ser entendido como um valor máximo.

Seguindo este raciocínio, a definição de \(avalia(a, n)\) envolve um
conjunto: todos os desempenhos para entradas de tamanho \(n\). Dado
\(n \in N\), considere \(D_n\) o conjunto das entradas como tamanho
\(n\):

\(D_n = \{ d \in D : tamanho(d) = n\}\).

No geral, o desempenho de um algoritmo depende {[}do tamanho{]} da
entrada. Assim, é razoável considerar o tamanho máximo e médio da
entrada.


    \subsubsection{Complexidade média (ou esperada)}


    Considere um algoritmo \(a\) possa receber 100 entradas \(d_j\) de
tamanho 10 (\(n = 10\)). Cada entrada \(d_j\) possui desempenho \(r_j\).
Assim, o desempenho esperado é o valor médio dado por:

\(c_M(a, n) = avalia_M(a, n) = \frac{r_1 + r_2 + ... + r_{100}}{100}\)

ou

\(avalia_M(a, n) = \frac{desempenho(a, d_1) + desempenho(a, d_2) + ... + desempenho(a, d_{100})}{100}\).

Em outras palavras, o desempenho esperado considera a média esperada dos
desempenhos do algoritmo. Neste caso, considera-se que cada uma das
\(d_j\) entradas possui a mesma probabilidade de ocorrer, ou seja, segue
uma \emph{distribuição uniforme}.

De maneira geral, a \emph{complexidade média (ou esperada)} de um
algoritmo é dada por:

\(avalia_M(a, n) = \sum_{d \in D_n}{} (prob(d) \times desempenho(a, d))\)

onde \(prob(d)\) dá a probabilidade de ocorrer a entrada \(d\) no
conjunto \(D_n\).

\textbf{Exercício 1.6}: Considere um algorimto \(a\) cujo desempenho
sobre cada entrada é o tamanho da entrada. Determine sua complexidade
média \(avalia_M(a, 50)\) considerando distribuição uniforme.


    \subsubsection{Complexidade pessimista (pior caso)}


    O algoritmo do \textbf{Exercício 1.6} possui uma característia
interessante: o desempenho do alogoritmo sobre cada entrada é o tamanho
da entrada. Assim, o seu pior desempenho sobre entradas com tamanho até
20 será 20:

\(avalia(a, [1, 2, ..., 20]) = {\mathrm{max}} \{1, 2, ..., 20\} = 20\)

A \emph{complexidade pessimista} de um algoritmo fornece o seu
desempenho no pior caso (ou o pior desempenho que se pode esperar) e é
definida como:

\(avalia_P(a, n) = \mathrm{max} \frac{desempenho(a, d) \in \mathrm(R)}{d \in D_n}\).

Em outras palavras, a \emph{complexidade pessimista} de um algoritmo
\(a\) é o valor máximo de seus desempenhos sobre todas as entradas com
tamanho \(n\).

\textbf{Exercício 1.7:} Considere o algoritmo abaixo:

\textbf{Algoritmo 1.2}: encontrar o máximo valor na lista

    \begin{Shaded}
\begin{Highlighting}[]
\NormalTok{n = }\FunctionTok{tamanho}\NormalTok{(lista);}
\NormalTok{max = lista[}\DecValTok{1}\NormalTok{];}
\KeywordTok{for}\NormalTok{(i = }\DecValTok{2}\NormalTok{; i < n; i++) \{}
    \KeywordTok{if} \NormalTok{(max < lista[i]) \{}
        \NormalTok{max = lista[i];}
    \NormalTok{\}}
\NormalTok{\}}
\KeywordTok{return} \NormalTok{max;}
\end{Highlighting}
\end{Shaded}

    Determine a \emph{complexidade pessimista} \(avalia_P(a, 20)\).

\textbf{Exercício 1.8:} Como poderia ser definida a \emph{complexidade
otimista}? Discuta a sua utilidade.

\textbf{Exercício 1.9:} Determine as complexidades média e pessimista do
\textbf{Algoritmo 1.1}.


    % Add a bibliography block to the postdoc
    
    
    
    \end{document}
